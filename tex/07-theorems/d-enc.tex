\subsection{Theorems on symmetric encryption schemes}
\paragraph{Constructing symmetric encryption schemes}
We describe how to build secure symmetric encryption schemes. We start with IND-CPA security
and then turn to AE-security. Since PRFs and PRPs tend to operate on blocks, we will often
need an \emph{encoding} scheme which encodes messages into a multiple of the block length.
\begin{definition}[Encoding scheme]
  We call two functions $\O{encode}_\lambda:\bin^*\rightarrow\bin^*$ and $\O{decode}_\lambda:\bin^*\rightarrow\bin^*$
  an \emph{encoding scheme} if
  \begin{itemize}
    \item $\O{encode}_\lambda$ and $\O{decode}_\lambda$ are computable in polynomial time polynomial in $\lambda$ and the length of the input.
    \item $\O{decode}_\lambda$ is the inverse of $\O{encode}_\lambda$, i.e., for all $x\in\bin^*$, we have
          $\O{decode}_\lambda(\O{encode}_\lambda(x))=x$. In particular, $\O{encode}_\lambda$ is \emph{injective}, i.e., if $x\neq x'$, then $\O{encode}_\lambda(x)\neq\O{encode}_\lambda(x')$.
    \item the length $\O{encode}_\lambda$ only depends on the length of the input, i.e., if $\abs{x}=\abs{x'}$, then
          $\abs{\O{encode}_\lambda(x)}=\abs{\O{encode}_\lambda(x')}$.
    \item for all $x\in\bin^*$, $\abs{\O{encode}_\lambda(x)}$ is divisible by $\lambda$.
  \end{itemize}
\end{definition}

\begin{theorem}[PRP in CBC is IND-CPA]
  Let $(\O{encode}_\lambda,\O{decode}_\lambda)$ be an encoding scheme and let $f$ be a secure $(\lambda,\lambda)$-secure pseudorandom permutation. Then, the
  following encryption scheme $\se_{\text{CBC-f}}$ is IND-CPA-secure.
  \begin{center}
    \begin{pchstack}
      \procedure{$\se_{\text{CBC-f}}.\O{enc}(k,m)$}{
      \lambda\gets\abs{k}\\
      m'\gets\O{encode}_{\lambda}(m)\\
      \pcvar{nonce}\sample\bin^\lambda\\
      \pcvar{c}^0\gets\pcvar{nonce}\\
      \ell\gets \frac{\abs{m'}}{\lambda}\\
      \pcfor i=1..\ell\\
      \pcind x^i\gets m'_{i\lambda..(i+1)\lambda-1}\\
      \pcind \pcvar{c}_i\gets f(k,x^i\oplus c^{i-1})\\
      c\gets (c_0,..,c_\ell)\\
      \pcreturn c}
      \pchspace
      \procedure{$\se_{\text{CBC-f}}.\O{dec}(k,c)$}{
      \lambda\gets\abs{k}\\
      \ell\gets \frac{\abs{c}}{\lambda}-1\\
      \text{Parse }c_0,..,c_\ell\gets c\\
      \pcfor i=1..\ell\\
      \pcind x^i\gets c^{i-1}\oplus f_{\text{inv}}(k,c^i)\\
      m'\gets x^1||..||x^\ell)\\
      m\gets \O{decode}_{\abs{k}}(m')\\
      \pcreturn m}
    \end{pchstack}
  \end{center}
\end{theorem}

\begin{theorem}[PRF in CTR is IND-CPA]
  Let $(\O{encode}_\lambda,\O{decode}_\lambda)$ be an encoding scheme and let $f$ be a secure $(\lambda,\lambda)$-secure pseudorandom function. Then, the encryption scheme $\se_{\text{CTR-f}}$ is IND-CPA-secure.
  \begin{center}
    \begin{pchstack}
      \procedure{$\se_{\text{CTR-f}}.\O{enc}(k,m)$}{
      \lambda\gets\abs{k}\\
      \pcvar{nonce}\sample\bin^\lambda\\
      m'\gets\O{encode}_\lambda(m)\\
      \ell\gets  \frac{\abs{m'}}{\lambda}\\
      \pcfor i=1..\ell\\
      \pcind \pcvar{pad}_i\gets f(k,\pcvar{nonce}+i)\\
      \pcvar{pad}'\gets\pcvar{pad}_1||..||\pcvar{pad}_\ell\\
      \pcvar{pad}\gets\pcvar{pad}_{1..\abs{m}}\\
      c\gets m \oplus \pcvar{pad}\\
      \pcreturn (\pcvar{nonce},c)}
      \pchspace
      \procedure{$\se_{\text{CTR-f}}.\O{dec}(k,(\pcvar{nonce},c))$}{
      \lambda\gets\abs{k}\\
      \ell\gets \frac{\abs{c}}{\lambda}\\
      \pcfor i=1..\ell\\
      \pcind \pcvar{pad}^i\gets f(k,\pcvar{nonce}+i)\\
      \pcvar{pad}'\gets\pcvar{pad}^1||..||\pcvar{pad}^\ell\\
      \pcvar{pad}\gets\pcvar{pad}_{1..\abs{c}}\\
      m'\gets c \oplus \pcvar{pad}\\
      m\gets\O{decode}_\lambda(m')\\
      \pcreturn m}
    \end{pchstack}
  \end{center}
\end{theorem}


\paragraph{Generic transformations on symmetric encryption schemes}
\begin{theorem}[IND-CPA]
  If $\se$ is an IND-CPA secure symmetric encryption scheme, then $\se_1$ is an IND-CPA secure symmetric encryption scheme.
  \begin{center}
    \begin{theorem}[AE]
      If $\se$ is an IND-CPA secure symmetric encryption scheme, and $\m$ is an UNF-CMA-secure message authentication code, then
      $\se_{\text{etm}}$ is an AE-secure symmetric encryption scheme.
    \end{theorem}

    \begin{pchstack}
      \procedure{$\se_1.\enc(k,m)$}{
        c'\sample\se.\enc(k,m)\\
        c \gets c'||1 \\
        \pcreturn c
      }
      \pcvspace
      \procedure{$\se_1.\dec(k,c)$}{
        c'\gets c[1..\abs{c}-1]\\
        m\gets\se.\dec(k,c')\\
        \pcreturn m
      }
      \pchspace
      \procedure{$\se_{\text{etm}}.\enc(k,m)$}{
        \pcparse (k_\m,k_\se)\gets k\\
        c'\sample\se.\enc(k_\se,m)\\
        \tau\sample\m.\mac(k_\m,c')\\
        c \gets (c',\tau) \\
        \pcreturn c
      }
      \pchspace
      \procedure{$\se_{\text{etm}}.\dec(k,c)$}{
        \pcparse (k_\m,k_\se)\gets k\\
        \pcparse (c',\tau)\gets c\\
        m\gets\se.\dec(k_\se,c')\\
        \pcif 1\gets\verify(k_\m,c',\tau)\ \pcreturn m\\
        \pcelse \pcreturn \bot
      }
    \end{pchstack}
  \end{center}

\end{theorem}