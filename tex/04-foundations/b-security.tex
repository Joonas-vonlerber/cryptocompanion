\subsection{Security}\label{ssec:security-foundations}
The concepts of a cryptographic system and an adversary are foundational to capturing security of cryptography. We already referred to adversaries and cryptographic systems in the statement at the very beginning of this section. So, now, let us start our discussion of security by elaborating on what we mean by the \emph{adversary} concept and by the concept of a \emph{cryptographic system}.

Cryptography studies the security of systems, functionalities and communication from adversarial interference. It is helpful to think of such situations as dividing agents into two groups: the system with its users, and the adversary. The system has some functionality that its users want to use. This functionality includes correctness (e.g., an encrypted messaging application shall transmit messages to the intended recipient), but it also includes some intuition of what it would mean to break the security of that functionality (e.g., unwanted access to the messages by third parties). It would be impossible to study the security of such a system without also considering the agent that tries to break the security. We call this agent the adversary. 
\begin{quote}
Cryptography studies the interplay between the system and the adversary.
\end{quote}


So far, our concepts have been rather general, since there are countless different types of security we might wish to study. The concepts are easier to understand with a concrete example, so let's quickly go over what the Transport Layer Security (TLS) protocol is, and what sort of security we want to define for it. The TLS protocol secures most of our internet communication. To simplify, the functionality of TLS is to send messages over the network \emph{securely}. But what does secure mean? In the case of TLS, there are three different security properties we want for this functionality. First, we want \emph{authentication}, meaning that we can be sure of the identity of the agent sending a message. Second, we want \emph{integrity}, meaning that the messages are not modified between being sent and being received. Thirdly, we want \emph{confidentiality}, meaning that the contents of the message can not be seen by anyone expect the intended recipient. These concepts are still quite vague. The aim of the course is to clarify what exactly these types of security guarantees mean, and how we can prove that they hold (and under which assumptions). But before we go into further details, let's discuss security definitions at a more general level.

\paragraph{Security Definitions} What is a good security definition? We can define any number of different security notions. Therefore, when putting forward a new security notion, we should explain why it is useful, why it captures what we aim for and how it differs from ones that have been defined before (if any). A good security definition (1) has a clear conceptual motivation, (2) has a clear description, (3) comes with a discussion of necessary, arbitrary and convenient choices as well as (4) a discussion of advantages/disadvantages compared to other definitions.

Security definitions are necessarily intertwined with a notion of an adversary. To clearly understand security notions, we also have to have a clear understanding of the adversary. The adversary is the agent that tries to break the security of a system. There are two things that go into breaking the security of a system: access to the system and what is done with that access. We model the \emph{adversary} as an algorithm that gets some inputs from the system and returns something. We call the inputs that the adversary gets the \emph{adversarial capabilities}. 
It depends on the situation being studied which capabilities the adversary has, and the capabilities have to be specified when defining a security notion. It is very common for the adversary to have the capability to read any messages sent over the network or even to entirely control the network traffic. There are also more specialized capabilities that an adversary might have, including being able to choose which messages a client sends (partial knowledge and/or control over the message content as, e.g., in the case of cookie-forwarding) or getting to see the outputs of a function for chosen inputs.

\begin{quote}
How should we choose the adversarial capabilities?
\end{quote}

An important insight (on which we elaborate below) is that we do not know what the exact capabilities of a real-life adversary are.
Cryptography has to protect against attacks that even the designer of the system has not thought of. 
Therefore, in our specifications of adversarial capabilities, we will typically \emph{over-approximate} a bit. 
For example, we say that a network adversary can drop, re-order, delay, re-route and replace all network messages.
In a real-life attack scenario, an adversary can typically only perform such actions partially. 
For example, the adversary might have only access to part of the network and might only be able to inject message but not be able to block all other messages. 
Unfortunately, capturing the \emph{exact} attack capabilities of various network adversaries might 
be impossible. 
So, if we manage to specify a \emph{bigger} set of adversarial capabilities that comprises all network attacks that might be carried out, then we can consider such a set of adversarial capabilities as \emph{sufficiently strong}. 
If we can show security against such a strong adversary, then, in particular, the protocols we build are secure also against weaker (realistic) adversaries which only have a subset of the capabilites. To summarize,
\begin{quote}
in a good model, the adversary's capabilities can be \emph{more} than in real-life, but \emph{should not be less}.
\end{quote}
This sentence is an excellent summary for decision-making for whether or not to include a certain capability into the model. I.e., it is always good to be on the side of allowing the adversary \emph{more} rather than less. However, this principle is a bit too simple on its own and needs to be complemented by a second principle. Consider, for example, the situation where a user is coerced into revealing its key to the adversary. In this case, the adversary can impersonate the user, send messages on their behalf and read all their communication. However, we might have a feeling that this is not actually an attack against the \emph{cryptographic system} itself, but rather an attack onto the circumstances in which the system is used. In any case, our cryptographic system cannot provide security in this situation and thus, we have to exclude this attack from the model. We call attacks against which \emph{no} cryptographic system can protect \emph{trivial attacks}, and the second principle for adversarial capabilites states that
\begin{quote}
in a good model, there are no trivial attacks.
\end{quote}

Similarly to being unable to estimate precisely the adversarial capabilities, we also do not know in advance what the adversary will do.
That's why we model the adversary as a general algorithm, without specifying what that algorithm does, and we would like to state that we provide security against any adversary as long as the adversary only uses the allowed adversarial capabilities (inputs from the system).
However, cryptography can typically be broken if an adversary invests \emph{an infeasible amount of time}, for example by trying out all possible keys.

\begin{quote}
Our models provide security against arbitrary \emph{efficient} adversaries.
\end{quote}
Moreover, cryptography can typically be broken with some very small probability, e.g., if the adversary guesses the key correctly.
So, when we say that \emph{a system is secure}, then this typically means that 

\begin{quote}
\emph{all efficient
adversaries} only have \emph{very small probability} in breaking the system. 
\end{quote}
The terms \emph{efficient} and \emph{small} can be interpreted in different ways. 
The interpretation this course uses is discussed in Section \ref{section:definitions}.

\paragraph{Proofs}
In addition to defining notions of security, we want to prove things about the security of specific systems.
Showing that an attack works against a system is typically easy: one executes the attack and thereby
convinces the observer that the attack succeeds. 
When we design secure systems, we want to do the opposite: we want to show that no attacks against the system exist. 
This is tremendously difficult, since there is an \emph{infinity} of attack strategies. 
How can we show that none of them work?
Even if we are precise about the adversary's attack capabilities, there are still an infinity of
strategies.
For example, a network adversary might insert various different messages---how can
we show that \emph{none} of these approaches work?
We would like to \emph{prove} that indeed, no such attack can succeed, and we show in this course
how to prove such a statement.

To be sure our proofs are correct, they need to be rigorous. 
To write rigorous proofs, we need precise, technical definitions of the things we study. 
So far, we have focused on describing our modeling choices, rather than technical accuracy.
Section \ref{section:definitions} will go over the technical details involved in writing security definitions, and Section \ref{section:primitives} will get to the work of actually writing security definitions.
To understand those technical details, it is useful to understand some of the conceptual steps we take to bridge the gap between the general remarks above and formal security definitions.

It is very hard to study a system\footnote{We would typically rather use the term \emph{protocol} rather than \emph{system}, but these terms have only vague boundaries and are not formally defined, and we can treat the term system as the most general term.} as complicated as TLS.
We therefore break it down into smaller parts, called \emph{primitives}.
A primitive should encapsulate some cryptographic functionality as simply as possible.
We can then break down the study of large cryptographic systems into two tasks: (1) studying the security of the primitives it uses, (2) studying how combining primitives affects the security of the system. (1) is what we are discuss here. Section \ref{ssec:composed-algorithms} discusses how we go about (2).

\paragraph{Security games}
The final conceptual step we want to introduce here is that of \emph{game-based definitions}.
This is a way of formalising the idea of an adversary breaking the security of a system.
The idea is to think of the cryptographic system being a game the adversary is playing.
The moves that the adversary is allowed to make are defined the adversarial capabilities, and the goal of the game is to break the security of the system.

In the case of a \emph{decision game}, the idea is that if the real system is impossible to distinguish from a system that functions perfectly, then the system is secure.
After all, the adversary being able to break the security of a system would certainly mean that they can notice they are not interacting with a unbreakable system. The technical details of this idea are presented in Section~\ref{ssec:security-games}.

A second type of security game is a \emph{search game}. Here, the task of the adversary is to produce an output of a specific type. Within the context of this course, we use this type of security less, but it is a very useful concept nevertheless. For example, it is intuitive to define \emph{integrity} as a search game where the adversaries task is to produce a message (also known as \emph{forgery}) that looks like it was sent by a user even though it wasn't. I.e., here, the adversary's task is to \emph{search} for such a message. At the same time, such search properties can also be encoded as a decision game. Namely, we can ask the adversary to distinguish between the real system (which would accept an adversary's forgery) and an (ideal) system that only accepts user-generated messages. The only way for the adversary to distinguish the real and the ideal system is by finding a forgery. For details on games, see Section~\ref{ssec:security-games}.

That finishes our discussion of the foundations of security. Much of the course is focused on fleshing out the details of the ideas discussed here, and applying them. However, there are some foundational concepts that aren't directly about cryptography, but are assumed by our treatment of cryptography. We don't spend much time on these things during the course, but they are needed for a bottom-up, complete understanding of cryptography. These topics deal mostly with computer science: how computation is modeled, what the runtime of algorithms is, and how algorithms are composed into larger systems. In addition to these, there is a section on probability, explaining how we use it in this course.