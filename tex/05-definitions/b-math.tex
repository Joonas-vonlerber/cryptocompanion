\subsection{Mathematical notation}\label{ssec:math-notation}
We start the section by defining some useful mathematical notation. First, we define some bare-bones notation used in algorithms and later definitions. Then we move on to describe the notation we use to describe probabilities.

When writing code, you often need data structures of some kind. The same is true of our discussion of cryptography. We use mathematical notation to define our data stuctures\footnote{The mathematical notation leaves out many details that would be important in real-life programming or in researching programming languages. However, these details are not important in the type of cryptography this course deals with.}. A \emph{set} is a collection of unique items. Sets can be defined by eumerating their members in curly braces: $\{0,1\}$ is the set containing 0 and 1. A \emph{table} is a collection of items which can be accessed by index. For a table $T$, we use the notation $T[n]$ to refer to the item of $T$ at index $n$ (see the next section for how we handle empty tables and sets). A \emph{string} is a sequence of bits. We use lower-case letters to signify variables. A \emph{function} is a deterministic mapping from some set $A$ to a set $B$, written $f: A \rightarrow B$.

The number $1$ is interperted as meaning \emph{true}, and $0$ as meaning \emph{false}. Logical operators can then be considered to be functions on bits.

There are some mathematical operations and shorthands we use. The most important are given in Table \ref{table:math}.
\begin{table}
\begin{tabular}{c | l}
Notation & Meaning \\ \hline
$A := b$ & $A$ is defined as being $b$ \\
$A \cup B$ & The set containing all elements of sets $A$ and $B$ \\
$A + x$ & Set containing all elements of $A$ and the new element $x$ \\
$\{x : P(x)\}$ & The set of all $x$ such that $P(x)$ is true \\
$a \in A$ & The element $a$ belongs to the set $A$ \\
$S^n$ & The set of strings that can be formed by picking a letter $n$ times from $S$ \\
$\bin^n$ & The set of bitstrings of length $n$ \\
$\bin^*$ & The set of all bitstrings \\
$\abs{S}$ & The number of elements in $S$ \\
$\abs{x}$ & The length of a string $x$ \\
$x||y$ & The concatenation of two strings $x$ and $y$ \\
$x_{abc\cdots}$ & String of bits of $x$ at positions $a$, $b$, $c$, etc. ($0110_1 = 0$, $0110_{13}=01$) \\
$x_{a \cdots b}$ & The substring of $x$ from position $a$ to $b$. ($0110_1 = 0$, $0110_{1 \cdots 3}=011$) \\
$x\oplus y$ & The bitwise XOR of two strings $x$ and $y$ ($0110 \oplus 1010 = 1100$) \\
$\binary(x)$ & The integer that the string $x$ represents in binary ($\binary(0011)$ = 3) \\ 
$\wedge$ & and ($0 \wedge 1 = 0$, $1 \wedge 1 = 1$) \\
$\vee$ & or ($0 \vee 1 = 1$, $1 \vee 1 = 1$)\\
$f: A \to B$ & The function $f$ maps values from $A$ to values in $B$\\
$A \times B$ & The set of all ordered pairs $(a,b)$ with $a \in A$, $b \in B$\\
\end{tabular}
\caption{Mathematical operations and shorthands}\label{table:math}
\end{table}

The above provide most of the mathematical notation we will use in this course. Don't worry if this is unfamiliar: you can always return here if you forget what some notation means. The first place we will use the above notation is in the following definitions for probability.

Let's start our definition of probability with an example. Take a function $f: \bin^n \rightarrow \bin$. Then the probability that ``$f(x)$ outputs 1 given a uniformly random bitstring of length n as input'' is written \(\probsub{x \sample \bin^n}{f(x) = 1}\). In the subscript $x \sample \bin^n$ we describe how the random input $x$ is chosen: it's taken from the set $\bin^n$, with every member of $\bin^n$ being equally likely to be chosen (this is what we mean by ``uniformly random''). 

To make our definition more general let's call the event we're interested in $P(x)$. Then the probability that $P(x)$ is true when $x$ is sampled uniformly at random from $S$ is \(\probsub{x \sample S}{P(x)}\). In the example above, $P(x)$ was the statement $f(x) = 1$ and $S$ the set $\bin^n$.

Now we have the notation in place, but how do we calculate what the probability is? We count how many possible inputs there are, then for how many of these the event occurs, and then we divide the two numbers (see Section \ref{foundations:probability} for a more detailed discussion). For a concrete example, let's take the probability \(\probsub{x \sample \bin^n}{f(x) = 1}\) from above, and let $f(x) := \binary(x)$, $n := 5$. With these choices of $f$ and $n$ in place, we have
\begin{align*}
  \probsub{x\sample \bin^n}{f(x)=1} & :=\frac{\abs{\{z:\  z\in\bin^n\ \wedge \ f(z)=1\}}}{\abs{\bin^n}} \\
                                    & =\frac{\abs{\{z:\  z\in\bin^n\ \wedge \ f(z)=1\}}}{2^n} \\
                                    & =\frac{1}{2^n} = \frac{1}{2^5} = \frac{1}{32} 
\end{align*}
In the calculations above, we used the fact that $\abs{\{z:\  z\in\bin^n\ \wedge \ f(z)=1\}}$ is equal to 1. This is the case since $f(x)$ is 1 only if $x$ is the bitstring $00001$.