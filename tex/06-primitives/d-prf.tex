\newpage
\subsection{Pseudorandom functions (PRF)}
If two parties share a random function, they can communicate securely. A random function is a function where for each input value, an output is drawn, uniformly at random. A \emph{pseudo}random function (PRF) is a keyed deterministic(!) function which looks like a random function to outsiders (who do not know the key). Here, outsiders are observers which can observe inputs and outputs of the function. For a deterministic function (with a public description, cf. \emph{Kerkoff Principle}) to be computationally indistinguishable, the deterministic function needs to rely on a secret, namely a secret key. Importantly, based on this key (of length $\lambda$), a pseudorandom function is able to emit a behaviour which looks like that of a random function although the description of the latter consists of exponentially many random bits. We will consider different kinds of PRFs. A ($\lambda$,$\lambda$)-PRF maps fixed inputs of length $\lambda$ to fixed-length outputs of length $\lambda$. A ($*$,$\lambda$)-PRF maps inputs of arbitrary length to fixed-length outputs of length $\lambda$. Finally, a ($\lambda$,$*$)-PRF maps inputs of fixed length $\lambda$ to an arbitrary output length, specified by the caller of the function. ($*$,$*$)-PRFs map inputs or arbitrary length $\lambda$ to an arbitrary output length, specified by the caller of the function. We encode all these PRF-variants into the same syntax by including a pair of parameters $(\pcvar{in},\pcvar{out})$, each of which can be $*$ or $\lambda$.

\begin{syntax}[($\pcvar{in}$,$\pcvar{out}$)-PRF syntax]
	Consider a pair $\pcvar{in},\pcvar{out}\in\{\lambda,*\}$. A ($\pcvar{in}$,$\lambda$)-PRF is a (deterministic) function
	\[f : \bin^*\times \bin^* \rightarrow \bin^*\]
	that can be computed in deterministic polynomial time, which satisfies the correctness criterion
	\[\forall \lambda\in\NN\,\forall k\in\bin^\lambda,x\in\bin^{\pcvar{in}}\;\abs{f(k,x)}=\lambda.\]
	A ($\pcvar{in}$,$*$)-PRF is a (deterministic) function
	\[f : \bin^*\times \bin^*\times \NN \rightarrow \bin^*\]
	that can be computed in deterministic polynomial time, which satisfies the correctness criterion
	\[\forall \lambda\in\NN\,\forall k,x\in\bin^{\pcvar{in}}\;\abs{f(k,x,L)}=L.\]
\end{syntax}

Note that $(\pcvar{in},*)$-PRFs take three inputs (with a length parameter $L$ as the last input) and $(\pcvar{in},\lambda)$-PRFs only
take two. This translates into the interface given to the adversary, since the adversary should also be allowed to
specify the output length for a $(\pcvar{in},*)$-PRFs. Thus, we provide different security definitions, the first one
for $(\pcvar{in},\lambda)$-PRFs and the second one for $(\pcvar{in},*)$-PRFs. In addition, we also provide modular versions
of the definitions.


\begin{security}[$(\pcvar{in},\lambda)$-PRF]
	Consider $\pcvar{in}\in\{\lambda,*\}$. Let the games $\M{Gprf}_f^0$, and $\M{Gprf}^1$ be defined as follows
	\begin{codebox}
		\begin{center}
			\begin{pchstack}
				\begin{pcvstack}
					\underline{\underline{$\M{Gprf}^0$}}\\
					\\
					\procedure{Package Parameters}{
						\lambda\text{:} \< \quad \text{security parameter}\\
						\pcvar{in}\text{:} \< \quad \text{input length}\;\lambda\text{ or }*\\
						f\text{:} \< \quad \text{function},\,(\pcvar{in},\lambda)\text{-PRF}
					}
					\pcvspace
					\procedure{Package State}{
						k\text{:} \< key
					}
					\pcvspace
					\procedure{$\O{EVAL}(x)$}{
						\pcassert x\in\bin^{\text{in}}\\
						\pcif k=\bot:\\
						\pcind k\sample\bin^\lambda\\
						y\gets f(k,x)\\
						\pcreturn y}
				\end{pcvstack}
				\pchspace
				\begin{pcvstack}
					\underline{\underline{$\M{Gprf}^1$}}\\
					\\
					\procedure{Package Parameters}{
						\lambda\text{:} \< \quad \text{security parameter}\\
						\pcvar{in}\text{:} \< \quad \text{input length}\;\lambda\text{ or }*\\
					}
					\pcvspace
					\procedure{Package State}{
						T\text{:} \< \quad \pckw{table}[\pctype{bitstring} \to \pctype{bitstring}]
					}
					\pcvspace
					\procedure{$\O{EVAL}(x)$}{
						\pcassert x\in\bin^{\text{in}}\\
						\pcif T[x] = \bot:\\
						\pcind T[x]\sample\bin^\lambda\\
						y\gets T[x]\\
						\pcreturn y}
				\end{pcvstack}
			\end{pchstack}
		\end{center}
	\end{codebox}
	\vspace{5mm}
	A $(\pcvar{in},\lambda)$-PRF $f$ is $(\pcvar{in},\lambda)$-PRF-secure if for all PPT adversaries $\adv$, the real game $\M{Gprf}_f^0$ and ideal game $\M{Gprg}^1$ are computationally indistinguishable, that is, the advantage
	\[\mathbf{Adv}_{\adv}^{
		\M{Gprf}_f^0,
		\M{Gprf}^1}
		(\lambda):=\abs{\prob{1=\adv\circ \M{Gprf}_f^0}-\prob{1=\adv\circ \M{Gprf}^1}}\]
	is negligible in $\lambda$.
\end{security}

\begin{remark}
	The table $T$ models a random function, since whenever it is called, it samples a fresh image value of length $\lambda$---unless the same input was queried before. In this case, it returns the same answer.
\end{remark}





\begin{security}[$(\pcvar{in},*)$-PRF]
	Consider $\pcvar{in}\in\{\lambda,*\}$. Let the games $\M{Glprf}_f^0$, and $\M{Glprf}^1$ be defined as follows
	\begin{codebox}
		\begin{center}
			\begin{pchstack}
				\begin{pcvstack}
					\underline{\underline{$\M{Glprf}^0$}}\\
					\\
					\procedure{Package Parameters}{
						\lambda\text{:} \< \quad \text{security parameter}\\
						\pcvar{in}\text{:} \< \quad \text{input length}\;\lambda\;\text{or}\;*\\
						\pcvar{out}\text{:} \< \quad \text{output length}*\\
						f\text{:} \< \quad \text{function},\,(\pcvar{in},*)\text{-PRF}
					}
					\pcvspace
					\procedure{Package State}{
						k\text{:} \< key
					}
					\pcvspace
					\procedure{$\O{EVAL}(x,L)$}{
						\pcassert x\in\bin^{\text{in}}\\
						\pcif k=\bot:\\
						\pcind k\sample\bin^\lambda\\
						y\gets f(k,x,L)\\
						\pcreturn y}
				\end{pcvstack}
				\pchspace
				\begin{pcvstack}
					\underline{\underline{$\M{Glprf}^1$}}\\
					\\
					\procedure{Package Parameters}{
						\lambda\text{:} \< \quad \text{security parameter}\\
						\pcvar{in}\text{:} \< \quad \text{input length}\;\lambda\;\text{or}\;*\\
						\pcvar{out}\text{:} \< \quad \text{output length}\;*\\
					}
					\pcvspace
					\procedure{Package State}{
						T\text{:} \< \quad \pckw{table}[\pctype{bitstring},\pctype{integer} \to \pctype{bitstring}]
					}
					\pcvspace
					\procedure{$\O{EVAL}(x,L)$}{
						\pcassert x\in\bin^{\text{in}}\\
						\pcif T[x,L] = \bot:\\
						\pcind T[x,L]\sample\bin^L\\
						y\gets T[x,L]\\
						\pcreturn y}
				\end{pcvstack}
			\end{pchstack}
		\end{center}
	\end{codebox}
	\vspace{5mm}
	A $(\pcvar{in},*)$-PRF $f$ is PRF-secure if the real game $\M{Glprf}_f^0$ and ideal game $\M{Glprg}^1$ are computationally indistinguishable, that is, for all PPT adversaries $\adv$, the advantage
	\[\mathbf{Adv}_{\adv}^{
		\M{Glprf}_f^0,
		\M{Glprg}^1}
		(\lambda)
		:=\abs{\prob{1=\adv\circ \M{Glprf}_f^0}-\prob{1=\adv\circ \M{Glprf}^1}}	\]
	is negligible in $\lambda$.
\end{security}


\begin{remark}
	The table $T$ models a random function, since whenever it is called, it samples a fresh image value of appropriate length---unless the same input was queried before. In this case, it returns the same answer.
\end{remark}